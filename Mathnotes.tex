\documentclass[11pt, a4paper]{article}
\usepackage[utf8]{inputenc}
\usepackage[ngerman]{babel}
\usepackage[T1]{fontenc}
\usepackage{amsmath}
\usepackage{amsfonts}
\usepackage{amssymb}
\usepackage{mathtools} %for better possibilities with matrices
\usepackage{graphicx}
\usepackage[left=20mm, right=20mm, top=20mm, bottom=20mm]{geometry} %set margin

\usepackage{paralist} %enumeration/lists
\usepackage{tikz} %graphs/diagrams
\usepackage{forest} % easier trees
\usepackage{array} %"tables" for mathmode
\usepackage{centernot} % \centernot \implies
\usepackage{multicol} %multiple text collums
\usepackage{fancyvrb} %verbatim with Tabs (\begin{Verbatim} \end{Verbatim})
	\fvset{tabsize=4}
\usepackage{xstring} %for \matrix{}, Stringmanipulation
\usepackage{listings} %for code blocks
\usepackage{color} %for colours
\usepackage{pdfpages} %for including external pdf pages (\includepdf[pages={-}]{file})
\usepackage{siunitx} %for stylised SI units (eg \SI{3}{\tera\hertz} = 3THz)

% Macros
\newcommand{\qed}{\(\square\)} %\qed
	%easier to write martices, can replace '&' with ',' and '\\' with ';'
\renewcommand{\matrix}[1]{
	\begingroup\expandarg
	\StrSubstitute{\noexpand#1}{, }{ & }[\result]
	\StrSubstitute\result{\noexpand;}{\noexpand\\}[\result]
	\begin{pmatrix}\result\end{pmatrix}
	\endgroup
}
